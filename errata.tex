% This is the errata document for the homotopy type theory book.

% This file supports two book sizes:
% - Letter size (8.5" x 11")
% - US Trade size (6" x 9")
%
% To activate one or the other, uncomment the appropriate font size in
% the documentclass below, and then one of the two page geometry incantations
%
% NOTE: The 6" x 9" format is only experimental. It will break the
% title page, for example.

\PassOptionsToPackage{table}{xcolor}

% DOCUMENT CLASS
\documentclass[
%
%10pt % for US Trade 6" x 9" book
%
11pt % for Letter size book
]{article}
\usepackage{etex} % We're running out of registers and dimensions, or some such

\newcounter{chapter}            % So that macros.tex doesn't choke

% PAGE GEOMETRY
%
% Uncomment one of these

% We make the page 40pt taller than the standard LaTeX book.

% OPTION 1: Letter
\usepackage[papersize={8.5in,11in},
            twoside,
            includehead,
            top=1in,
            bottom=1in,
            inner=0.75in,
            outer=1.0in,
            bindingoffset=0.35in]{geometry}

% OPTION 2: US Trade
% \usepackage[papersize={6in,9in},
%             twoside,
%             includehead,
%             top=0.75in,
%             bottom=0.75in,
%             inner=0.5in,
%             outer=0.75in,
%             bindingoffset=0.35in]{geometry}

% HYPERLINKING AND PDF METADATA

\usepackage[pagebackref,
            colorlinks,
            citecolor=darkgreen,
            linkcolor=darkgreen,
            unicode,
            pdfauthor={Univalent Foundations Program},
            pdftitle={Homotopy Type Theory: Univalent Foundations of Mathematics},
            pdfsubject={Mathematics},
            pdfkeywords={type theory, homotopy theory, univalence axiom}]{hyperref}

% OTHER PACKAGES

% Use this package and stick \layout somewhere in the text to see
% page margins, text size and width etc. Useful for debugging page format.
%\usepackage{layout}

%%% Because Germans have umlauts and Slavs have even stranger ways of mangling letters
\usepackage[utf8]{inputenc}

%%% For table {tab:theorems}
\usepackage{pifont}

%%% Multi-Columns for long lists of names
\usepackage{multicol}

%%% Set the fonts
\usepackage{mathpazo}
\usepackage[scaled=0.95]{helvet}
\usepackage{courier}
\linespread{1.05} % Palatino looks better with this

\usepackage{graphicx}
\usepackage{comment}

\usepackage{wallpaper} % For the background image on the cover page

\usepackage{fancyhdr} % To set headers and footers

\usepackage{nextpage} % So we can jump to odd-numbered pages

\usepackage{amssymb,amsmath,amsthm,stmaryrd,mathrsfs,wasysym}
\usepackage{enumitem,mathtools,xspace}
\usepackage{xcolor} % For colored cells in tables we need \cellcolor
\usepackage{booktabs} % For nice tables
\usepackage{array} % For nice tables
\usepackage{supertabular} % For index of symbols
\definecolor{darkgreen}{rgb}{0,0.45,0}
\usepackage{aliascnt}
\usepackage[capitalize]{cleveref}
\usepackage[all,2cell]{xy}
\UseAllTwocells
\usepackage{natbib}
\usepackage{braket} % used for \setof{ ... } macro
\usepackage{tikz}
\usetikzlibrary{decorations.pathmorphing}

\usepackage{etoolbox}           % hacking commands for TOC

\usepackage{mathpartir}         % for formal.tex appendix, section 3

\usepackage[numbered]{bookmark} % add chapter/section numbers to the toc in the pdf metadata

\input{macros}

%%%% Indexing
\usepackage{makeidx}
\makeindex

%%%% Header and footers
\pagestyle{fancyplain}
\setlength{\headheight}{15pt}
\renewcommand{\sectionmark}[1]{\markright{\textsc{\thesection\ #1}}}

\lhead[\fancyplain{}{{\thepage}}]%
      {\fancyplain{}{\nouppercase{\rightmark}}}
\rhead[\fancyplain{}{\nouppercase{\leftmark}}]%
      {\fancyplain{}{\thepage}}
\cfoot{\textsc{\scriptsize \textcolor{gray}{[Draft of \today]}}}
\lfoot[]{}
\rfoot[]{}

%%%% Chapter & part style
\usepackage{titlesec}
\titleformat{\part}[display]{\fontsize{40}{40}\fontseries{m}\fontshape{sc}\selectfont}{\hfil\partname\ \Roman{part}}{20pt}{\fontsize{60}{60}\fontseries{b}\fontshape{sc}\selectfont\hfil}
\titleformat{\chapter}[display]{\fontsize{23}{25}\fontseries{m}\fontshape{it}\selectfont}{\chaptertitlename\ \thechapter}{20pt}{\fontsize{35}{35}\fontseries{b}\fontshape{n}\selectfont}

\input{main.labels}

\usepackage{longtable}

\title{Errata for the HoTT Book, first edition}

\begin{document}
\maketitle

\noindent
\begin{longtable}{llp{10.5cm}}
  \textbf{Location} & \textbf{Fixed in} & \textbf{Change} \\ \hline \endhead
  \autoref{sec:types-vs-sets}
  & 154-g42698c2
  & Clarify that algorithmic decidability of judgmental equality is only meta-theoretic.\\
  %
  \autoref{sec:universes}
  & 42-g4bc5cc2
  & Cumulativity means some elements do not have unique types, the index $i$ on $\UU_i$ is not an internal natural number, and typical ambiguity must be justified by reinserting indices.\\
  %
  \autoref{sec:universes,sec:pi-types}
  & 42-ga34b313
  & Explain that we can't define $\Fin$ and $\fmax$ yet where we first mention them.\\
  %
  \autoref{sec:pi-types}
  & 43-g1e9baad
  & Add $\mathsf{swap}$ as another example of a polymorphic function.\\
  %
  \autoref{rmk:introducing-new-concepts}
  & 80-g8f95fa5
  & In the discussion of formation rules, the dependent function type example should be $\prd{x:A} B(x)$.\\
  %
  \autoref{sec:finite-product-types}
  & 51-g67e86db
  & Better explanation of recursion on product types, why it is justified, and how it relates to the uniqueness principle.\\
  %
  \autoref{sec:sigma-types}
  & 2-gbe277a8
  & In the types of $g$ and $\ind{\sm{x:A}B(x)}$, there is a $\prd{a:A}{b:B(x)}$ in which $x$ should be $a$.\\
  %
  \autoref{sec:sigma-types}
  & 27-gd0bfa0d
  & At two places in the definition of $\ac$, $R(a,\fst(g(x)))$ should be $R(x,\fst(g(x)))$.\\
  %
  \autoref{sec:sigma-types}
  & 125-g7fdadbf
  & When substituting $\lam{x} \fst(g(x))$ for $f$ while verifying that $\ac$ is well-typed, the left side of the judgmental equality should be $\tprd{x:A} R(x,\fst(g(x)))$, not $\tprd{x:A} R(x,\fst(f(x)))$.\\
  %
  \autoref{sec:coproduct-types}
  & 30-g264d934
  & In two displayed equations, $f(\inl(b))$ should be $f(\inr(b))$.\\
  %
  \autoref{sec:type-booleans}
  & 125-g433f87e
  & In the definition of binary products in terms of $\bool$, the definitions of $\fst(p)$ and $\snd(p)$ should be switched to match the order of arguments to $\rec\bool$ and $\ind\bool$.\\
  % 
  \autoref{sec:pat}
  & 111-g1e868fa
  & When translating English to type theory, ``unnamed variables'' are unnamed in English but must be named in type theory.\\
  %
  \autoref{ex:iterator}
  & 78-gcce4dc0
  & The second defining equation of $\ite$ should have right-hand side $c_s(\ite(C,c_0,c_s,n))$.\\
  %
  \autoref{ex:ackermann}
  & 110-gfe4641b
  & To match the usual Ackermann--P\'eter function, the second displayed equation should be $\ack(\suc(m),0) \jdeq \ack(m,1)$.\\
  %
  \autoref{sec:compute-cartprod}
  & 74-g9896e32
  & In the type of $\pairpath$ (just after the proof of \autoref{thm:path-prod}), the second factor in the domain should be $\id{\proj{2}(x)}{\proj{2}(y)}$.\\
  %
  \autoref{thm:trans-prod}
  & 76-ga42354c
  & The third displayed judgmental equality in the proof should be $\transfib{B}{p}{\proj{2}x} \jdeq \proj2x$.\\
  %
  \autoref{sec:compute-coprod}
  & 101-g645f763
  & In \autoref{thm:path-coprod} and the preceding paragraph, in the equivalence $\eqv{(\inl(a)=x)}{\code(x)}$, the variable $a$ should be $a_0$. \\
  %
  \autoref{subsec:prop-subsets}
  & 86-g39feab1
  & The definition of subset containment should say $\prd{x:A}(P(x)\rightarrow Q(x))$, not $\fall{x:A}(P(x)\Rightarrow Q(x))$, as the latter notation has not been introduced yet.\\
  %
  \autoref{thm:retract-contr}
  & 95-gce0131f
  & In the proof, $p$ should be $r$ to match the preceding definition of retraction.\\
  %
  \autoref{lem:qinv-autohtpy}
  & 87-g693e9b9
  & At the end of the proof, \autoref{thm:contr-paths} should be cited as the reason why $\sm{g:A\to A} (g = \idfunc[A])$ is contractible.\\
  %
  \autoref{sec:univalence-implies-funext}
  & 114-gaba76c8
  & The point of \autoref{UA-eqv-hom-eqv} is that it follows from univalence without assuming function extensionality separately.\\
  %
  \autoref{sec:w-types}
  & 125-g433f87e
  & Use $\bfalse$ for $0$ and $\btrue$ for $\suc$, to match the ordering of $\bfalse$ and $\btrue$ in \autoref{sec:type-booleans}.\\
  %
  \autoref{thm:identity-systems}
  & 139-gd5c5d01
  & In the proof of \ref{item:identity-systems4}$\Rightarrow$\ref{item:identity-systems1}, the type of $D'$ should be $(\sm{b:A} R(b)) \to \type$.\\
  %
  \autoref{sec:dependent-paths}
  & 54-gd4a47c2
  & Soon after \autoref{rmk:defid}, the phrase ``An element $b:P(\base)$ in the fiber over the constructor $\base:\nat$'' should say $\base:\Sn^1$.\\
  %
  \autoref{thm:isaprop-isofhlevel}
  & 101-g713f48c
  & The base case in the proof is just \autoref{thm:isprop-iscontr}.\\
  %
  \autoref{ex:s2-colim-unit}
  & 101-ga366be2
  & ``entires'' should be ``entirely''.\\
  %
  \autoref{thm:freudcode}
  & 87-g3f977b2
  & In the second displayed equation in the proof, $\merid(x_1)$ should be $\opp{\merid(x_1)}$.\\
  %
  \autoref{thm:freudlemma}
  & 88-g0c0be67
  & The type of $m$ should be $a_1=a_2$, the second display should begin with $C(a_1,\transfib{B}{\opp m}{b})$, and the proof should say ``we may assume $a_2$ is $a_1$ and $m$ is $\refl{a_1}$''.\\
  % 
  \autoref{thm:ordord}
  & 140-g55de417
  & The second sentence of the proof should say ``By well-founded induction on $A$, suppose $\ordsl A b$ is accessible for all $b<a$.''\\
  %
  \autoref{thm:ordunion}
  & 140-gd7f8960
  & The statement should say $X:\UU$ rather than $X:\UU_\UU$.\\
  %
  \autoref{thm:wellorder}
  & 140-gcca0bcf
  & The penultimate sentence of the proof should say ``if $a<b$ and $b<c$'' rather than ``if $a<b$ and $a<c$''.\\
  %
  \autoref{lem:untruncated-linearity-reals-coincide}
  & 87-g82b27c3
  & \eqref{eq:untruncated-linearity} should be $c:\prd{q, r : \Q} (q < r) \to (q < x) + (x < r)$, and therefore the use of $c$ in the proof should be $c(s,t)$ rather than $c(x,s,t)$.\\
  %
  \autoref{subsec:contexts}
  & 64-g7c2312e
  & Clarify the distinction between typing judgments and context well-formedness judgments, and
  remove the $\vdash$ from the notation for the latter.\\
  %
  \autoref{sec:more-formal-sigma}
  & 26-gcd691e8
  & In $\Sigma$-\comp\ and the following paragraph, $y.C$ should be $z.C$, and ``we bind \dots $y$ in $C$'' should likewise say $z$.\\
\end{longtable}

\end{document}
